\nonstopmode{}
\documentclass[oneside,final,14pt]{extreport}
\usepackage[T2A]{fontenc}
\usepackage[utf8]{inputenc}
\usepackage[english,ukrainian]{babel}
\usepackage{vmargin}
\setpapersize{A4}
\setmarginsrb{2cm}{1.5cm}{1cm}{1.5cm}{0pt}{0mm}{0pt}{13mm}
\usepackage{indentfirst}

\usepackage{microtype}

\usepackage{seqsplit}

% \usepackage[hyphens]{url}
\usepackage[colorlinks=false, pdfborder={0 0 0}]{hyperref}

\sloppy

\linespread{1.3}

\usepackage{acronym}

\begin{document}

% \chapter{Технічне завдання для дипломного проекту}
% \section{Найменування і область застосування}
% \section{Підстави для розробки}
% \section{Мета і призначення розробки}
% \section{Джерела розробки}
% \section{Технічні вимоги}
% \subsection{Вимоги до розроблюваного продукту}
% \subsection{Вимоги до програмного забезпечення}
% \section{Етапи розробки}

\tableofcontents

\chapter*{Список скорочень}
\addcontentsline{toc}{chapter}{Список скорочень}
\begin{acronym}[RTOSA]
  \acro{CPU}{Central Processing Unit}
  \acro{FPU}{Floating Point Unit}
  \acro{MMU}{Memory Management Unit}
  \acro{RAM}{Random Access Memory}
  \acro{ROM}{Read-Only Memory}
  \acro{RTOS}{Real-Time Operating System}
  \acro{SLOB}{Simple List of Blocks}
  \acro{SoC}{System on Chip}
\end{acronym}
% \printacronyms[heading=none]{}

\chapter{Огляд існуючих рішень}
\section{Embedded Linux}

Ядро Linux\cite{linux} пропонує широкий набір функціоналу, серед якого можна знайти безліч драйверів, підтримка великої кількості файлових систем, широкий вибір планувальників і алокаторів пам'яті. Однак, для зниження системних вимог, доводиться відмовлятися від більшості функцій, що надає ядро Лінукс.

Ядро Linux було портовано на велику кількість архітектур, в число яких входять такі як ARM, AVR32, MIPS, Xtensa. Ці архітектури використовуються у вбудованих системах. Однак Linux висуває досить високі вимоги до ресурсів системи. Мінімальний обсяг оперативної пам'яті, що вимагає система, дорівнює 4 мегабайтам, також Лінукс вимагає наявності \ac{MMU}, що перешкоджає його використанню в системах на нижній границі потужностей.

Незважаючи на це, Том Занусі модифікував ядро Лінукс і портував його на Intel Galileo\cite{linux-galileo}. Ця плата базована на мікроконтроллері Intel Quark SoC X1000, що має архітектуру Pentium і 512 кілобайт вбудованої оперативної пам'яті\cite{intel-galileo}.

Ядро розповсюджується під вільною ліцензією GPLv2.

\section{FreeRTOS}

Насьогодні, FreeRTOS\cite{freertos} є найпопулярнішою операційною системою реального часу для вбудованих систем\cite{freertos:popular}. Її було портовано на 35 мікроконтроллерів, що мають наступні архітектури: ARM, AVR, HCS12, MicroBlaze, Cortus, MSP430, PIC, Renesas~ H8/S, SuperH, RX, x86, 8052, Coldfire, V850, 78K0R, Fujitsu MB91460, Nios~ II, TMS570, RM4x.

Це дуже проста операційна система, що складається з чотирьох файлів на мові C. FreeRTOS надає методи для роботи з потоками, м'ютексами і семафорами, таймерами і чергами. Підтримується так званий tickless режим роботи (у цьому режимі переривання таймера відбуваються при нерівних інтервалах, і лише в міру необхідності).

Доступні чотири алокатори пам'яті:
\begin{itemize}
  \item Виділення пам'яті без можливості її звільнення.
  \item Простий алгоритм виділення і звільнення пам'яті без об'єднання вільних ділянок.
  \item Більш складний і швидкий алгоритм виділення і звільнення пам'яті з об'єднанням вільних ділянок.
  \item Обгортки навколо стандартної бібліотеки, що забезпечують взаємне виключення.
\end{itemize}

Ядро розповсюджується під модифікованою вільною ліцензією GPLv2\cite{freertos:license}. Модифікація дозволяє залишати закритим код додатків, що використовують FreeRTOS і розповсюджуються у вигляді виконуваного коду, а також забороняє використання FreeRTOS у еталонних теста. Також доступна комерційна підтримка і ліцензія під ім'ям OpenRTOS\cite{openrtos}.

\section{VxWorks}

VxWorks\cite{vxworks} це операційна система реального часу зозроблена як власницьке програмне забезпечення Wind River. VxWorks призначена для вбудованих систем реального часу, що вимагають детермінованої продуктивності, і часто, проходження сертифікації безпеки і захищенності.

VxWorks підтримує архітектури Intel (x86), MIPS, PowerPC, SH-4 і ARM, і може працювати в режимах симетричної, асиметричної і змішаному режимах багатопроцесорної обробки.

\begin{thebibliography}{0}
  \newcommand{\eresource}[2]{#1. [Електронний ресурс]. --- Режим доступу: \url{#2}}

  \bibitem{linux}
    \eresource{Linux.com | The source for Linux information}{https://linux.com/}
  \bibitem{linux-galileo}
    \eresource{microYocto and the Internet of Tiny}{http://events.linuxfoundation.org/sites/events/files/slides/tom.zanussi-elc2014.pdf}
  \bibitem{intel-galileo}
    \eresource{Galileo Feature Sheet}{http://www.intel.com/content/dam/support/us/en/documents/galileo/sb/galileo\_datasheet\_329681\_003.pdf}

  \bibitem{freertos}
    \eresource{FreeRTOS - Market leading RTOS (Real Time Operating System) for embedded systems with Internet of Things extensions}{http://www.freertos.org/}
  \bibitem{freertos:popular}
    \eresource{Android, FreeRTOS top EE Times' 2013 embedded survey}{http://www.eetimes.com/document.asp?doc\_id=1263083}
  \bibitem{freertos:license}
    \eresource{FreeRTOS License}{http://www.freertos.org/license.txt}
  \bibitem{openrtos}
    \eresource{OPENRTOS, part of embedded FreeRTOS -- OpenRTOS -- SafeRTOS family}{http://www.highintegritysystems.com/openrtos/}

  \bibitem{vxworks}
    \eresource{VxWorks}{http://windriver.com/products/vxworks/}
\end{thebibliography}

\end{document}
