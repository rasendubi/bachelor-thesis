\nonstopmode{}
\documentclass[oneside,14pt,a4paper,final]{extreport}
\usepackage[english,ukrainian]{babel}
\usepackage{fontspec}
\usepackage{polyglossia}

\defaultfontfeatures{Ligatures={TeX}}
\setmainfont{Times New Roman}
\setsansfont{Comic Sans MS}
\setmonofont{Courier New}

%\newfontfamily\ukrainianfont{Times New Roman}
\newfontfamily\cyrillicfont{Times New Roman}
\newfontfamily\cyrillicfonttt{Courier New}

\setmainlanguage{ukrainian}
\setotherlanguage{english}

\usepackage{indentfirst}

% single space between sentences
\frenchspacing

\usepackage[hyphens]{url}
\usepackage[colorlinks=false, pdfborder={0 0 0}, linktocpage=true]{hyperref}

\usepackage{microtype}

\sloppy

\linespread{1.3}

\usepackage{acronym}

\usepackage{titlesec}
\titleformat{\chapter}[hang]{\filcenter}{\bfseries\MakeUppercase{\chaptertitlename} \thechapter. }{0pt}{\bfseries\MakeUppercase}{}
\titleformat{\section}{\normalsize\bfseries}{\thesection}{1em}{}
\titleformat{\subsection}{\normalsize\bfseries}{\thesubsection}{1em}{}
%\titlespacing*{\chapter}{0pt}{-30pt}{8pt}
% these were {*4}{*4}
\titlespacing*{\section}{\parindent}{*2}{*2}
\titlespacing*{\subsection}{\parindent}{*2}{*2}

\usepackage{geometry}
\geometry{left=3cm}
\geometry{right=1.5cm}
\geometry{top=2.4cm}
\geometry{bottom=2.4cm}

\usepackage{tocloft}

\usepackage{etoolbox}
\makeatletter
\patchcmd{\l@chapter}{#1}{\MakeUppercase{#1}}{}{}
\makeatother

\renewcommand{\cfttoctitlefont}{\hspace{0.38\textwidth} \bfseries\MakeUppercase}
\renewcommand{\cftbeforetoctitleskip}{-1em}
\renewcommand{\cftaftertoctitle}{\vspace{-2.5em}}
%\renewcommand{\cftaftertoctitle}{\mbox{}\hfill \\ \mbox{}\hfill{\footnotesize Стр.}\vspace{-2.5em}}
\renewcommand{\cftchapfont}{}
% \renewcommand{\cftchapleader}{\bfseries\cftdotfill{\cftchapdotsep}}
\renewcommand{\cftchappresnum}{\MakeUppercase{\chaptername} }
\renewcommand{\cftchapaftersnum}{.\space}
\renewcommand{\cftchapaftersnumb}{}

\newlength\mylength
\settowidth\mylength{\cftchappresnum\cftchapaftersnum\space}
\addtolength{\cftchapnumwidth}{\mylength}

%\renewcommand{\cftsecfont}{\hspace{31pt}}
%\renewcommand{\cftsubsecfont}{\hspace{11pt}}
%\renewcommand{\cftbeforechapskip}{1em}
\renewcommand{\cftparskip}{-1mm}
\renewcommand{\cftdotsep}{1}
\setcounter{tocdepth}{2}

\usepackage[nottoc]{tocbibind}
\makeatletter
\renewcommand\@biblabel[1]{#1.}
\makeatother

\textwidth=175mm
\textheight=260mm
\oddsidemargin=-.4mm
\headsep=5mm

\topmargin=-1in
\unitlength=1mm

%\usepackage{xltxtra}

% https://bitbucket.org/fat_angel/opengostfont/downloads
\newfontfamily{\GostA}[AutoFakeSlant=0.4,Path=fonts/,Numbers=Uppercase,Scale=MatchLowercase,Ligatures=NoCommon]{OpenGostTypeA-Regular}
\newfontfamily{\GostB}[AutoFakeSlang=0.4,Path=fonts/,Numbers=Uppercase,Scale=0.90]{OpenGostTypeB-Regular}

\def\VL{\line(0,1){15}}
\def\HL{\line(1,0){185}}
\def\Box#1#2{\makebox(#1,5){#2}}
\def\simpleGrad{\sl\small\noindent\hbox to 0pt{%
    \vbox to 0pt{%
      \noindent\begin{picture}(185,287)(5,0)
      \linethickness{0.3mm}
      \put(0,0){\framebox(185,287){}}
      \put(0,0){\Box{7}{\GostB Лит.}}
      \put(0,15)\HL
      \multiput(0,5)(0,5){2}{\line(1,0){65}}
      \put(7,0){\VL\Box{10}{\GostB Изм.}}
      \put(17,0){\VL\Box{23}{\GostB\No~докум.}}
      \put(40,0){\VL\Box{15}{\GostB Подп.}}
      \put(55,0){\VL\Box{10}{\GostB Дата}}
      \put(65,0){\VL\makebox(110,15){\large\sc код}}
      \put(175,0){\VL\makebox(10,10){\GostA\normalsize\thepage}}
      \put(175,10){\line(1,0){10}}
      \end{picture}
    }
  }
}

\makeatletter
\def\@oddhead{\simpleGrad}
\def\@oddfoot{}
\makeatother

\begin{document}

\renewcommand\bibname{Список використаних джерел}

% \chapter{Технічне завдання для дипломного проекту}
% \section{Найменування і область застосування}
% \section{Підстави для розробки}
% \section{Мета і призначення розробки}
% \section{Джерела розробки}
% \section{Технічні вимоги}
% \subsection{Вимоги до розроблюваного продукту}
% \subsection{Вимоги до програмного забезпечення}
% \section{Етапи розробки}

\tableofcontents

\chapter*{Список скорочень}
\addcontentsline{toc}{chapter}{Список скорочень}
\begin{acronym}[TCP/IPA]
  \acro{CAN}{Controller Area Network}
  \acro{CPU}{Central Processing Unit}
  \acro{DOSFS}{FAT-based file system}
  \acro{FPU}{Floating Point Unit}
  \acro{HRFS}{High Reliability File System}
  \acro{MMU}{Memory Management Unit}
  \acro{NFS}{Network File System}
  \acro{RAM}{Random Access Memory}
  \acro{ROM}{Read-Only Memory}
  \acro{RTOS}{Real-Time Operating System}
  \acro{SLOB}{Simple List of Blocks}
  \acro{SoC}{System on Chip}
  \acro{TCP/IP}{Transport Control Protocol/Internet Protocol}
  \acro{USB}{Universal Serial Bus}
\end{acronym}
% \printacronyms[heading=none]{}

\chapter{Огляд існуючих рішень}
\section{Embedded Linux}

Ядро Linux\cite{linux} пропонує широкий набір функціоналу, серед якого можна знайти безліч драйверів, підтримка великої кількості файлових систем, широкий вибір планувальників і алокаторів пам'яті. Однак, для зниження системних вимог, доводиться відмовлятися від більшості функцій, що надає ядро Лінукс.

Ядро Linux було портовано на велику кількість архітектур, в число яких входять такі як ARM, AVR32, MIPS, Xtensa. Ці архітектури використовуються у вбудованих системах. Однак Linux висуває досить високі вимоги до ресурсів системи. Мінімальний обсяг оперативної пам'яті, що вимагає система, дорівнює 4 мегабайтам, також Лінукс вимагає наявності \ac{MMU}, що перешкоджає його використанню в системах на нижній границі потужностей.

Незважаючи на це, Том Занусі модифікував ядро Лінукс і портував його на Intel Galileo\cite{linux-galileo}. Ця плата базована на мікроконтроллері Intel Quark SoC X1000, що має архітектуру Pentium і 512 кілобайт вбудованої оперативної пам'яті\cite{intel-galileo}.

Ядро розповсюджується під вільною ліцензією GPLv2.

\section{VxWorks}

VxWorks\cite{vxworks} це операційна система реального часу зозроблена як власницьке програмне забезпечення Wind River. VxWorks призначена для вбудованих систем реального часу, що вимагають детермінованої продуктивності, і часто, проходження сертифікації безпеки і захищенності.

VxWorks підтримує архітектури Intel x86, MIPS, PowerPC, Freescale ColdFire, Intel i960, SPARC, Fujitsu FR-V, SH-4, ARM, StrongARM і xScale, і може працювати y режимах симетричної, асиметричної і змішаної багатопроцесорної обробки. Операційна система включає у себе багатозадачне ядро з витісняючим планувальником, засоби міжпроцесорної взаємодії і синхронізації, стек протоколів Bluetooth, \acs{TCP/IP}, \acs{USB}, \acs{CAN}, а також файлові системи \ac{HRFS}, \acs{DOSFS}, \ac{NFS}.

Мінімальний обсяг оперативної пам'яті дорівнює 20-ти кілобайтам.

VxWorks активно використовується в аерокосміній, захисній, автомобільній і робототехнічній галузях. Відомими прикладами використання є марсоходи Sojourner, Spirit, Opportunity\cite{vxworks:rovers}, загальна система ядра Boeing 787 Dreamliner\cite{vxworks:boeing}.

\section{FreeRTOS}

Насьогодні, FreeRTOS\cite{freertos} є найпопулярнішою операційною системою реального часу для вбудованих систем\cite{freertos:popular}. Її було портовано на 35 мікроконтроллерів, що мають наступні архітектури: ARM, AVR, HCS12, MicroBlaze, Cortus, MSP430, PIC, Renesas~ H8/S, SuperH, RX, x86, 8052, Coldfire, V850, 78K0R, Fujitsu MB91460, Nios~ II, TMS570, RM4x.

Це дуже проста операційна система, що складається з чотирьох файлів на мові C. FreeRTOS надає методи для роботи з потоками, м'ютексами і семафорами, таймерами і чергами. Підтримується так званий tickless режим роботи (у цьому режимі переривання таймера відбуваються при нерівних інтервалах, і лише в міру необхідності).

Доступні чотири алокатори пам'яті:
\begin{itemize}
  \item Виділення пам'яті без можливості її звільнення.
  \item Простий алгоритм виділення і звільнення пам'яті без об'єднання вільних ділянок.
  \item Більш складний і швидкий алгоритм виділення і звільнення пам'яті з об'єднанням вільних ділянок.
  \item Обгортки навколо стандартної бібліотеки, що забезпечують взаємне виключення.
\end{itemize}

FreeRTOS має надзвичайно низькі вимоги до оперативної пам'яті. Планувальник вимагає 236 байт оперативної пам'яті, плюс 76 байти для кожної черги і 64 байти для кожної задачі.

Ядро розповсюджується під модифікованою вільною ліцензією GPLv2\cite{freertos:license}. Модифікація дозволяє залишати закритим код додатків, що використовують FreeRTOS і розповсюджуються у вигляді виконуваного коду, а також забороняє використання FreeRTOS у еталонних теста. Також доступна комерційна підтримка і ліцензія під ім'ям OpenRTOS\cite{openrtos}.

\section{QNX Neutrino}

QNX\cite{qnx} це комерційна Unix-подібна операційна система спрямована в першу чергу на вбудовані системи. Операційна система має мікроядерну архітектуру. QNX Neutrino була портована на велику кількість платформ і працює на Intel 8088, x86, MIPS, PowerPC, SH-4, ARM, StrongARM, XScale.

Ядро QNX містить лише планувальник задач, міжпроцесну взаємодію, перенаправлення переривань і таймери. Все інше виконується як процеси рівня користувача, включаючи керування пам'яттю.

QNX має закриту ліцензію, але також має окрему ліцензію для академічного і некомерційного використання\cite{qnx:noncommercial}.

% \section{Contiki}

\begin{thebibliography}{00}
  \newcommand{\eresource}[2]{#1. [Електронний ресурс]. --- Режим доступу: \url{#2}}

  \bibitem{linux}
    \eresource{Linux.com | The source for Linux information}{https://linux.com/}
  \bibitem{linux-galileo}
    \eresource{microYocto and the Internet of Tiny}{http://events.linuxfoundation.org/sites/events/files/slides/tom.zanussi-elc2014.pdf}
  \bibitem{intel-galileo}
    \eresource{Galileo Feature Sheet}{http://www.intel.com/content/dam/support/us/en/documents/galileo/sb/galileo\_datasheet\_329681\_003.pdf}

  \bibitem{freertos}
    \eresource{FreeRTOS - Market leading RTOS (Real Time Operating System) for embedded systems with Internet of Things extensions}{http://www.freertos.org/}
  \bibitem{freertos:popular}
    \eresource{Android, FreeRTOS top EE Times' 2013 embedded survey}{http://www.eetimes.com/document.asp?doc\_id=1263083}
  \bibitem{freertos:license}
    \eresource{FreeRTOS License}{http://www.freertos.org/license.txt}
  \bibitem{openrtos}
    \eresource{OPENRTOS, part of embedded FreeRTOS -- OpenRTOS -- SafeRTOS family}{http://www.highintegritysystems.com/openrtos/}

  \bibitem{vxworks}
    \eresource{VxWorks}{http://windriver.com/products/vxworks/}
  \bibitem{vxworks:rovers}
    \eresource{Inside NASA's Curiosity: It's an Apple Airport Extreme... with wheels | ExtremeTech}{http://www.extremetech.com/extreme/134041-inside-nasas-curiosity-its-an-apple-airport-extreme-with-wheels}
  \bibitem{vxworks:boeing}
    \eresource{787 Dreamliner (Common Core Systems) | Customers | AdaCore}{http://www.adacore.com/customers/787-dreamliner-common-core-system/}

  \bibitem{qnx}
    \eresource{QNX Neutrino RTOS}{http://www.qnx.com/products/neutrino-rtos/neutrino-rtos.html}
  \bibitem{qnx:noncommercial}
    \eresource{QNX Neutrino Realtime Operating System}{http://www.qnx.com/download/download.html?dlc=proc&newsearch=yes&searchme=non-commercial&p=1&sort=bydate&sorttype=desc&searchdate=alltime}
\end{thebibliography}

\end{document}
