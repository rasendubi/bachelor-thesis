\documentclass[main.tex]{subfiles}

\begin{document}

\begin{specialpage}
  \begin{spacing}{1.5}
  \vfill

  \textbf{АНОТАЦІЯ}\\
  \vspace{5mm}
  \begin{minipage}{\textwidth}
    Даний дипломний проект присвячений розробці мови і компілятора для вбудованих пристроїв.
    В першому розділі було виконано аналіз предметної області, аналогів та конкурентів.
    У другому розділі було сформульовано вимоги до розроблюваної мови, проаналізовано найбільш типові програмні помилки, проведено проектування мови.
    У третьому розділі наведено опис компонентів реалізованого компілятора і його особливості.
  \end{minipage}

  \vfill

  % \vspace{1cm}
  \textbf{АННОТАЦИЯ}\\
  \vspace{5mm}
  \begin{minipage}{\textwidth}
    Данный дипломный проект посвящен разработке языка и компилятора для встроенных устройств.
    В первом разделе был выполнен анализ предметной области, аналогов и конкурентов.
    Во втором разделе были сформулированы требования к разрабатываемому языку, проанализированы найболее типичные програмные ошибки, произведено проектирование языка.
    В третьем разделе приведено описание компонентов реализированого компилятора и его особенности.
  \end{minipage}

  \vfill

  %\vspace{1cm}
  \textbf{ABSTRACT}\\
  \vspace{5mm}
  \begin{minipage}{\textwidth}
    This thesis project is dedicated to development of a programming language and a compiler for embedded devices.
    The first chapter contains domain analysis, comparison of analogs and competitors.
    The second chapter describes requirements to the language, analyses most typical programming errors, and contains the language design.
    The third chapter describes the compiler components.
  \end{minipage}
  \vfill

  \end{spacing}
\end{specialpage}

\end{document}
