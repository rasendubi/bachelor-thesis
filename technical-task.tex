\documentclass[main.tex]{subfiles}

\begin{document}

\makeatletter
\renewcommand\thesection{\@arabic\c@section.}
\makeatother

\titleformat\section[block]{\bfseries\filcenter}{\thesection}{1em}{}

\renewcommand\stamppartname{Технічне завдання}
\newpart{tt}{ІАЛЦ.467100.002 ТЗ}

\tableofcontents

\clearpage
%\tableofcontents
%\let\tableofcontents\relax

% \chapter{Технічне завдання для дипломного проекту}
\section{Найменування та сфера застосування}

Технічне завдання розроблено для написання дипломного проекту на тему <<\stampname>>. Сферою застосування є вбудовані операційні системи та операційні системи реального часу. Розроблена система може використовуватися в промисловості.

\section{Обгрунтування розробки}

Основою для розробки є завдання на здобуття ступеня <<бакалавр програмної інженерії>>, що затверджено кафедрою обчислювальної техніки Національного Технічного Університету України <<Київського Політехнічного Інституту>>.

\section{Мета та призначення розробки}
Метою роботи є створення системи для швидкої обробки сигналів на пристроях, що мають менше кілобайта оперативної пам'яті.

\section{Джерела розробки}
Вхідними даними для роботи є теоретичні матеріали з теми <<Операційні системи>>, <<Вбудовані системи>>, публікації в періодичних виданнях, статті в Інтернеті за необхідною тематикою.

\section{Технічні вимоги}
Основні вимоги до устаткування:
\begin{itemize}
\item Плата STM32F4Discovery.
\end{itemize}

\section{Вимоги до програмного забезпечення}
Програмне забезпечення повинно бути відсутнім.

\section{Етапи розробки}

\noindent
Затвердження теми роботи \hfill 15.11.2015--15.12.2015 \\
Вивчення та аналіз завдання \hfill 10.01.2016--25.01.2016 \\
Розробка архітектури та загальної структури системи \hfill 26.01.2016--10.02.2016 \\
Розробка структур окремих підсистем \hfill 11.02.2016--14.03.2016 \\
Програмна реалізація системи \hfill 15.03.2016--19.04.2016 \\
Оформлення пояснювальної записка \hfill 20.04.2016--20.05.2016 \\

\finalizepart{}

\end{document}
