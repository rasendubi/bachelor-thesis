\documentclass[main.tex]{subfiles}

\begin{document}

%\subfile{technical-task-title}

\makeatletter
\renewcommand\thesection{\@arabic\c@section.}
\makeatother

\titleformat\section[block]{\bfseries\filcenter}{\thesection}{1em}{}

\renewcommand\stamppartname{Технічне завдання}
\newpart{tt}{ІАЛЦ.467100.002 ТЗ}

%\tableofcontents

\clearpage
\tableofcontents
%\let\tableofcontents\relax

% \chapter{Технічне завдання для дипломного проекту}
\section{Найменування та сфера застосування}

Технічне завдання розроблено для написання дипломного проекту на тему <<\thesistitle{}>>. Сферою застосування є вбудовані системи, операційні системи реального часу, системні утіліти. Розроблена мова може використовуватися в промисловості.

\section{Обгрунтування розробки}

Підставою для розробки є завдання на дипломне проектування, затверджене кафедрою обчислювальної техніки Національного технічного університету України <<Київський політехнічний інститут імені Сікорського>> (НТУУ <<КПІ ім. Сікорського>>).

\section{Мета та призначення розробки}
Метою роботи є створення мови для пришвидшення розробки низькорівневого програмного забезпечення, забезпечення його надійності і захисту.

\section{Джерела розробки}
Вхідними даними для роботи є теоретичні матеріали з теми <<Мови програмування>>, <<Вбудовані системи>>, публікації в періодичних виданнях, статті в Інтернеті за необхідною тематикою.

\section{Вимоги до програмного забезпечення}
Розроблена мова має:
\begin{itemize}[nosep]
\item бути придатною для реалізації систем реального часу
\item підтримувати компіляцію в найбільш розповсюджені комп'ютерні архітектури
\item мінімізувати кількість програмних помилок, які може допустити програміст
\end{itemize}

\section{Етапи розробки}

\noindent
  \begin{tabular}{| m{5mm} | m{80mm} | m{40mm} | m{20mm} |}
    \hline
    \No\newline з/п & Назва етапів виконання дипломного проекту & Термін виконання етапів проекту & Примітка\\
    \hline
    1 & Вивчення рекомендованої літератури & 04.09.2017 &\\ \hline
    2 & Аналіз існуючих методів розв'язання задачі & 10.09.2017 &\\ \hline
    3 & Постановка та формалізація задачі & 15.09.2017 &\\ \hline
    4 & Аналіз вимог до програмного забезпечення & 17.09.2017 &\\ \hline
    5 & Розробка програмного забезпечення & 10.11.2017 &\\ \hline
    6 & Оформлення пояснювальної записки & 01.11.2017 &\\ \hline
  \end{tabular}
%Вивчення рекомендованої літератури \hfill 04.09.2017\\
%Аналіз існуючих методів розв'язання задачі \hfill 10.09.2017\\
%Постановка та формалізація задачі \hfill 15.09.2017\\
%Аналіз вимог до програмного забезпечення \hfill 17.09.2017\\
%Моделювання програмного забезпечення \hfill 21.09.2017\\
%Розробка програмного забезпечення \hfill 10.11.2017\\
%Оформлення пояснювальної записки \hfill 01.11.2017\\
%Подання ДП на попередній захист \hfill 16.12.2017\\
%Подання ДП на основний захист \hfill 18.12.2017\\

\finalizepart{}

\end{document}
